\documentclass[]{tufte-book}

\hypersetup{colorlinks}% uncomment this line if you prefer colored hyperlinks (e.g., for onscreen viewing)

%%
% For graphics / images
\usepackage{graphicx}
\setkeys{Gin}{width=\linewidth,totalheight=\textheight,keepaspectratio}
\graphicspath{{graphics/}}
\usepackage{hyperref}
\usepackage{chemformula}

%%
% Book metadata
\title[SCINE ReaDuct manual]{User Manual \vskip 0.5em {\setlength{\parindent}{0pt} \Huge SCINE ReaDuct 1.0.0}}
\author[The SCINE ReaDuct Developers]{The SCINE ReaDuct Developers: \newline \noindent Christoph Brunken, Jan Unsleber, Alain Vaucher, and Markus Reiher}
\publisher{ETH Z\"urich}

%%
% If they're installed, use Bergamo and Chantilly from www.fontsite.com.
% They're clones of Bembo and Gill Sans, respectively.
%\IfFileExists{bergamo.sty}{\usepackage[osf]{bergamo}}{}% Bembo
%\IfFileExists{chantill.sty}{\usepackage{chantill}}{}% Gill Sans

%\usepackage{microtype}

%%
% Just some sample text
\usepackage{lipsum}

%%
% For nicely typeset tabular material
\usepackage{booktabs}

% The fancyvrb package lets us customize the formatting of verbatim
% environments.  We use a slightly smaller font.
\usepackage{fancyvrb}
\fvset{fontsize=\normalsize}

%%
% Prints argument within hanging parentheses (i.e., parentheses that take
% up no horizontal space).  Useful in tabular environments.
\newcommand{\hangp}[1]{\makebox[0pt][r]{(}#1\makebox[0pt][l]{)}}

%%
% Prints an asterisk that takes up no horizontal space.
% Useful in tabular environments.
\newcommand{\hangstar}{\makebox[0pt][l]{*}}

%%
% Prints a trailing space in a smart way.
\usepackage{xspace}

%%
% Some shortcuts for Tufte's book titles.  The lowercase commands will
% produce the initials of the book title in italics.  The all-caps commands
% will print out the full title of the book in italics.
\newcommand{\vdqi}{\textit{VDQI}\xspace}
\newcommand{\ei}{\textit{EI}\xspace}
\newcommand{\ve}{\textit{VE}\xspace}
\newcommand{\be}{\textit{BE}\xspace}
\newcommand{\VDQI}{\textit{The Visual Display of Quantitative Information}\xspace}
\newcommand{\EI}{\textit{Envisioning Information}\xspace}
\newcommand{\VE}{\textit{Visual Explanations}\xspace}
\newcommand{\BE}{\textit{Beautiful Evidence}\xspace}

\newcommand{\TL}{Tufte-\LaTeX\xspace}

% Prints the month name (e.g., January) and the year (e.g., 2008)
\newcommand{\monthyear}{%
  \ifcase\month\or January\or February\or March\or April\or May\or June\or
  July\or August\or September\or October\or November\or
  December\fi\space\number\year
}


% Prints an epigraph and speaker in sans serif, all-caps type.
\newcommand{\openepigraph}[2]{%
  %\sffamily\fontsize{14}{16}\selectfont
  \begin{fullwidth}
  \sffamily\large
  \begin{doublespace}
  \noindent\allcaps{#1}\\% epigraph
  \noindent\allcaps{#2}% author
  \end{doublespace}
  \end{fullwidth}
}

% Inserts a blank page
\newcommand{\blankpage}{\newpage\hbox{}\thispagestyle{empty}\newpage}

\usepackage{units}

% Typesets the font size, leading, and measure in the form of 10/12x26 pc.
\newcommand{\measure}[3]{#1/#2$\times$\unit[#3]{pc}}

% Macros for typesetting the documentation
\newcommand{\hlred}[1]{\textcolor{Maroon}{#1}}% prints in red
\newcommand{\hangleft}[1]{\makebox[0pt][r]{#1}}
\newcommand{\hairsp}{\hspace{1pt}}% hair space
\newcommand{\hquad}{\hskip0.5em\relax}% half quad space
\newcommand{\TODO}{\textcolor{red}{\bf TODO!}\xspace}
\newcommand{\ie}{\textit{i.\hairsp{}e.}\xspace}
\newcommand{\eg}{\textit{e.\hairsp{}g.}\xspace}
\newcommand{\na}{\quad--}% used in tables for N/A cells
\providecommand{\XeLaTeX}{X\lower.5ex\hbox{\kern-0.15em\reflectbox{E}}\kern-0.1em\LaTeX}
\newcommand{\tXeLaTeX}{\XeLaTeX\index{XeLaTeX@\protect\XeLaTeX}}
% \index{\texttt{\textbackslash xyz}@\hangleft{\texttt{\textbackslash}}\texttt{xyz}}
\newcommand{\tuftebs}{\symbol{'134}}% a backslash in tt type in OT1/T1
\newcommand{\doccmdnoindex}[2][]{\texttt{\tuftebs#2}}% command name -- adds backslash automatically (and doesn't add cmd to the index)
\newcommand{\doccmddef}[2][]{%
  \hlred{\texttt{\tuftebs#2}}\label{cmd:#2}%
  \ifthenelse{\isempty{#1}}%
    {% add the command to the index
      \index{#2 command@\protect\hangleft{\texttt{\tuftebs}}\texttt{#2}}% command name
    }%
    {% add the command and package to the index
      \index{#2 command@\protect\hangleft{\texttt{\tuftebs}}\texttt{#2} (\texttt{#1} package)}% command name
      \index{#1 package@\texttt{#1} package}\index{packages!#1@\texttt{#1}}% package name
    }%
}% command name -- adds backslash automatically
\newcommand{\doccmd}[2][]{%
  \texttt{\tuftebs#2}%
  \ifthenelse{\isempty{#1}}%
    {% add the command to the index
      \index{#2 command@\protect\hangleft{\texttt{\tuftebs}}\texttt{#2}}% command name
    }%
    {% add the command and package to the index
      \index{#2 command@\protect\hangleft{\texttt{\tuftebs}}\texttt{#2} (\texttt{#1} package)}% command name
      \index{#1 package@\texttt{#1} package}\index{packages!#1@\texttt{#1}}% package name
    }%
}% command name -- adds backslash automatically
\newcommand{\docopt}[1]{\ensuremath{\langle}\textrm{\textit{#1}}\ensuremath{\rangle}}% optional command argument
\newcommand{\docarg}[1]{\textrm{\textit{#1}}}% (required) command argument
\newenvironment{docspec}{\begin{quotation}\ttfamily\parskip0pt\parindent0pt\ignorespaces}{\end{quotation}}% command specification environment
\newcommand{\docenv}[1]{\texttt{#1}\index{#1 environment@\texttt{#1} environment}\index{environments!#1@\texttt{#1}}}% environment name
\newcommand{\docenvdef}[1]{\hlred{\texttt{#1}}\label{env:#1}\index{#1 environment@\texttt{#1} environment}\index{environments!#1@\texttt{#1}}}% environment name
\newcommand{\docpkg}[1]{\texttt{#1}\index{#1 package@\texttt{#1} package}\index{packages!#1@\texttt{#1}}}% package name
\newcommand{\doccls}[1]{\texttt{#1}}% document class name
\newcommand{\docclsopt}[1]{\texttt{#1}\index{#1 class option@\texttt{#1} class option}\index{class options!#1@\texttt{#1}}}% document class option name
\newcommand{\docclsoptdef}[1]{\hlred{\texttt{#1}}\label{clsopt:#1}\index{#1 class option@\texttt{#1} class option}\index{class options!#1@\texttt{#1}}}% document class option name defined
\newcommand{\docmsg}[2]{\bigskip\begin{fullwidth}\noindent\ttfamily#1\end{fullwidth}\medskip\par\noindent#2}
\newcommand{\docfilehook}[2]{\texttt{#1}\index{file hooks!#2}\index{#1@\texttt{#1}}}
\newcommand{\doccounter}[1]{\texttt{#1}\index{#1 counter@\texttt{#1} counter}}

%attempt to allow footnotes in verbatim
\usepackage{verbatim}
\newcommand{\vfchar}[1]{%
  % the usual trick for using a "variable" active character
  \begingroup\lccode`~=`#1 \lowercase{\endgroup\def~##1~}{%
    % separate the footnote mark from the footnote text
    % so the footnote mark will occupy the same space as
    % any other character
    \makebox[0.5em][l]{\footnotemark}%
    \footnotetext{##1}%
  }%
  \catcode`#1=\active
}
\newenvironment{fverbatim}[1]
 {\verbatim\vfchar{#1}}
 {\endverbatim}


% Generates the index
\usepackage{makeidx}
\makeindex

%\usepackage{natbib}
\setcitestyle{numbers,square}

\usepackage{parskip}



\begin{document}

\setlength{\parindent}{0pt}

% Front matter
\frontmatter


% r.3 full title page
\maketitle


% v.4 copyright page
\newpage
\begin{fullwidth}
~\vfill
\thispagestyle{empty}
\setlength{\parindent}{0pt}
\setlength{\parskip}{\baselineskip}
Copyright \copyright\ \the\year\ \thanklessauthor

%\par\smallcaps{Published by \thanklesspublisher}

\par\smallcaps{https://scine.ethz.ch/download/readuct}

\par Unless required by applicable law or agreed to in writing, the software 
is distributed on an \smallcaps{``AS IS'' BASIS, WITHOUT
WARRANTIES OR CONDITIONS OF ANY KIND}, either express or implied. \index{license}

%\par\textit{First printing, \monthyear}
\end{fullwidth}

% r.5 contents
\tableofcontents

%\listoffigures

%\listoftables


%%
% Start the main matter (normal chapters)
\mainmatter

\let\cleardoublepage\clearpage
\chapter{Introduction}

The SCINE project requires stable algorithms for the refinement of elementary-reaction paths and associated transition-state 
structures. The SCINE \textsc{ReaDuct} module was designed to serve this purpose and can be driven from SCINE \textsc{Interactive} 
and SCINE \textsc{Chemoton}. However, as with all SCINE modules it is a stand-alone program that can drive standard quantum 
chemical software.

SCINE \textsc{ReaDuct} is a command-line tool that allows to carry out structure optimizations, transition state searches
and intrinsic reaction coordinate (IRC) calculations among other things.
For these calculations, it relies on a backend program to provide the necessary quantum chemical properties (such
as nuclear gradients). Currently, SCINE \textsc{Sparrow}\cite{sparrow} and \textsc{Orca}\cite{orca} are supported as backend programs.

In this manual, we describe the installation of the software, an example calculation as a hands-on 
introduction to the program, and the most import functions and options.\footnote{Throughout this manual, the most 
import information is displayed in the main text, whereas useful additional information is given as a side note like this one.}
A prospect on features in future releases and references for further reading are added at the end of this manual.\enlargethispage{\baselineskip}



\chapter{Obtaining the Software}
\label{ch:obtain}

\textsc{ReaDuct}  is distributed as open source software in the framework of the SCINE project (\href{https://scine.ethz.ch/}{www.scine.ethz.ch}).
Visit our website (\href{https://scine.ethz.ch/download/readuct}{www.scine.ethz.ch/download/readuct}) to obtain the software. 


\section{System Requirements}

\textsc{ReaDuct} can be used on any computer with a 64-bit x86 architecture. The software itself has only modest requirements
regarding the hardware performance. However, the underlying quantum-chemical calculations might become resource intensive 
if extremely large systems are studied. We advise to first explore the software with the fast semiempirical methods provided 
in \textsc{ReaDuct}. This allows one to quickly understand what to expect from the software rather than being confused by 
possibly long times waiting for more invovled quantum chemical calculations to finish.



\chapter{Installation}\label{ch:installation}

\textsc{ReaDuct} is distributed as an open source code. In order to compile \textsc{ReaDuct} from this source code, you need
\begin{itemize}
 \item a C++ compiler supporting the C++14 standard (we recommend gcc 7.3.0),
 \item cmake (we recommend version 3.9.0),
 \item the Boost library (we recommend version 1.64.0), and
 \item the Eigen3 library (we recommend version 3.3.2).
\end{itemize}
In order to compile the software, either directly clone the repository with git or extract the downloaded tarball, change 
to the source directory and execute the following steps:
\begin{verbatim}
git submodule init
git submodule update
mkdir build install
cd build
cmake -DCMAKE_BUILD_TYPE=Release -DBUILD_SPARROW=ON -DCMAKE_INSTALL_PREFIX=../install ..
make
make test
make install
export SCINE_MODULE_PATH=<source code directory>/install/lib
export PATH=$PATH:<source code directory>/install/bin
\end{verbatim}
This will configure everything, compile your software, run the tests, and install the software 
into the folder ``install''. Finally, it will add the \textsc{ReaDuct} binary to your \texttt{PATH} such that you can use
it without having to specify its full location. In this last command, you have to replace \texttt{<source code directory>}
with the full path where you stored the source code of \textsc{ReaDuct}.

In case you need support with the setup of \textsc{ReaDuct}, please contact us by writing to \href{scine@phys.chem.ethz.ch}{scine@phys.chem.ethz.ch}.



\chapter{Using the Standalone Binary}

\textsc{ReaDuct} is a command-line-only binary; there is no graphical user interface. Therefore, you always work with the
\textsc{ReaDuct} binary on a command line such as the Gnome Terminal or KDE Konsole.

All functionality is accessed via an input file following the YAML syntax. 


\section{General Structure of the Input File}

The general structure of a \textsc{ReaDuct} input file is as follows:

\begin{verbatim}
systems:
  - name: [system name]
    path: [path to coordinates file]
    program: [program name]
    method: [method name]
    settings:
      [settings key]: [settings value]
      ...

tasks:
  - type: [task type name]
    input: [input system name]
    output: [output system name]
    settings:
      [settings key]: [settings value]
      ...
 
\end{verbatim}

There are two major blocks, namely a \texttt{systems} block and a \texttt{tasks} block. You can define multiple systems
in the \texttt{systems} block and multiple tasks in the \texttt{tasks} block (see also section \nameref{sec:task_chaining}).

A system is a combination of nuclear coordinates (given as an XYZ file) and a calculation program (such as SCINE \textsc{Sparrow}
or ORCA) and method (such as PM6). Depending on the program and method used, different settings (such as molecular charge, spin
multiplicity, and convergence thresholds) can be given. A task specifies that a certain calculation type (such as a
structure optimization) should be carried out with a given (input) system. Different tasks can have different settings.
For every task, an output system can be assigned to be used in further tasks (for instance, the output system of a
structure optimization task contains the optimized nuclear coordinates).

For example, in order to do a simple structure optimization, you can use the following input file:

\begin{verbatim}
systems:
  - name: 'water'
    path: 'h2o.xyz'
    program: 'Sparrow'
    method: 'PM6'
    settings:
      molecular_charge: 0
      spin_multiplicity: 1

tasks:
  - type: 'geoopt'
    input: ['water']
    output: ['water_opt']
    settings:
      optimizer: 'lbfgs'
\end{verbatim}

This specifies a system named \texttt{water}, the nuclear coordinates are given by the XYZ file \texttt{h2o.xyz}. Any 
calculation performed on this system will use the PM6 method provided by SCINE \textsc{Sparrow}. For this system, a
structure optimization will be carried out; the structure will be optimized with the L-BFGS algorithm\cite{lbfgs}.


\section{Supported Programs and Methods}

\subsection{SCINE \textsc{Sparrow}}

SCINE \textsc{Sparrow} is fully supported by SCINE \textsc{ReaDuct}. If built with the cmake option \texttt{-DBUILD\_SPARROW=ON}
as described in section~\nameref{ch:installation}, it will be automatically downloaded and integrated into \textsc{ReaDuct}
at compile time.

In order to use SCINE \textsc{Sparrow} with \textsc{ReaDuct}, specify \texttt{program: 'Sparrow'} in the respective system
block and the desired calculation method (such as \texttt{'PM6'}) in the \texttt{method} key. All options supported by \textsc{Sparrow} 
can be defined in the settings block. See the \textsc{Sparrow} manual for a complete list of these options (the option 
names are identical to the command line option names of the \textsc{Sparrow} standalone binary).

\subsection{ORCA}

\textbf{Important note:} Support for ORCA\cite{orca} is currently not fully tested. There might be specific calculation
types and/or settings which do not work. Also, we cannot guarantee compatibility with any ORCA version different from
4.1.0 since we have no control over the output format of an external program. If you encounter any problems when
using ORCA together with \textsc{ReaDuct}, please write a short message to \href{scine@phys.chem.ethz.ch}{scine@phys.chem.ethz.ch}.

In order to use ORCA with \textsc{ReaDuct}, specify \texttt{program: 'OSUtils'} and \texttt{method: 'ORCA'} in 
the respective system block. You can specify the following settings in the settings block:
\begin{itemize}
\item \texttt{molecular\_charge}: This specifies the molecular charge. It can take on values between -10 and 10; by default,
it is zero.
\item \texttt{spin\_multiplicity}: This specifies the spin multiplicity. It can take on values between 1 and 10; by default,
it is 1.
\item \texttt{orca\_method}: This specifies the method string. By default, it is \texttt{PBE def2-SVP}, \textit{i.e.,} a
DFT calculation with the PBE exchange--correlation functional and the def2-SVP basis set is carried out. You can specify
any valid ORCA method string (see the ORCA manual for a complete list).
\item \texttt{self\_consistence\_criterion}: The threshold to which the electronic energy should be converged (given in
hartree). By default, it is 10\textsuperscript{$-$6}\,hartree.
\item \texttt{orca\_nprocs}: The number of processors to use in the ORCA calculations. By default, it is one, \textit{i.e.},
a serial calculation is carried out. Note that you have to specify the full ORCA binary path in case you want to do a
parallel calculation (see below).
\item \texttt{orca\_binary\_path}: This is used to specify the path to the ORCA binary. By default, this is set to
``orca'', \textit{i.e.}, this option need not be specified if ORCA is in your path and you want to do a serial
calculation (\texttt{orca\_nprocs: 1}). For a parallel calculation, you have to specify the full (absolute) path to
your ORCA binary here.
\item \texttt{orca\_filename\_base}: This specifies the basic filename (prefix) used for all files related to the ORCA calculations.
By default, it is set to ``orca\_calc''; therefore, the generated input file will be named ``orca\_calc.inp''.
\item \texttt{base\_working\_directory}: This specifies the directory in which the files for the ORCA calculations will
be stored. By default, this is set to the current directory. For each ORCA calculation a new directory will be
created inside the directory specified by \texttt{base\_working\_directory} to keep the files related to that specific
calculation.
\end{itemize}


\section{Tasks}

\subsection{Single Point Calculation}

The single point task can be used to obtain the electronic energy of a given system. In order to carry out this task,
specify any of the following in the respective task block: \texttt{type: 'single\_point'}, \texttt{type: 'singlepoint'},
\texttt{type: 'sp'}, or \texttt{type: 'energy'}.

\subsection{Hessian Calculation}

This task calculates the Hessian of a given system and outputs the vibrational frequencies. In order to carry out this task,
specify any of the following in the respective task block: \texttt{type: 'hessian'}, \texttt{type: 'frequency\_analysis'},
\texttt{type: 'frequencyanalysis'}, \texttt{type: 'frequencies'}, \texttt{type: 'frequency'}, or \texttt{type: 'freq'}.

\subsection{Structure Optimization}

This task is used in order to optimize the structure of a given system to a minimum on the potential energy surface. In 
order to carry out this task, specify any of the following in the respective task block: \texttt{type: 'geometry\_optimization'}, 
\texttt{type: 'geometryoptimization'}, \texttt{type: 'geoopt'}, or \texttt{type: 'opt'}.

The task works without the specification of any additional settings; the default settings work usually fine. However,
if desired, the following settings can always be set:
\begin{itemize}
\item \texttt{optimizer}: This sets the desired optimization algorithm. You can set \texttt{'lbfgs'} for the L-BFGS algorithm,
\texttt{'steepestdescent'} or \texttt{'sd'} for a steepest descent algorithm, and \texttt{'newtonraphson'} or \texttt{'nr'} for
a Newton--Raphson algorithm. By default, it is set to \texttt{'lbfgs'}.
\item \texttt{convergence\_step\_max\_coefficient}: The convergence threshold for the maximum absolute element of the last step taken.
By default set to \texttt{1.0e-4}.
\item \texttt{convergence\_step\_rms}: The convergence threshold for the root mean square of the last step taken. By default set to 
\texttt{5.0e-4}.
\item \texttt{convergence\_gradient\_max\_coefficient}: The convergence threshold for the maximum absolute element of the gradient. 
By default set to \texttt{5.0e-5}.
\item \texttt{convergence\_gradient\_rms}: The convergence threshold for the root mean square of the gradient. By default set to 
\texttt{1.0e-5}.
\item \texttt{convergence\_delta\_value}: The convergence threshold for the change in the functional value. By default set to
\texttt{1.0e-7}.
\item \texttt{convergence\_max\_iterations}: The maximum number of iterations. By default set to \texttt{100}.
\item \texttt{convergence\_requirement}: The number of criteria that have to converge besides the value criterion. This 
has to be between \texttt{0} and \texttt{4}; by default it is set to \texttt{3}.
\end{itemize}

If you specified \texttt{optimizer: 'lbfgs'}, you can also set the following options:
\begin{itemize}
\item \texttt{lbfgs\_maxm}: The number of parameters and gradients from previous iterations to keep. By default set to 
\texttt{50}.
\item \texttt{lbfgs\_linesearch}: Whether to use a line search or not. By default set to \texttt{true}.
\item \texttt{lbfgs\_c1}: The first parameter of the Wolfe conditions. This option is only relevant if line search is
used (see above). By default set to \texttt{0.0001}.
\item \texttt{lbfgs\_c2}:  The second parameter of the Wolfe conditions. This option is only relevant if line search is
used (see above). By default set to \texttt{0.9}.
\item \texttt{lbfgs\_step\_length}: The initial step length. By default set to \texttt{1.0}.
\item \texttt{lbfgs\_use\_trust\_radius}: Whether to use the trust radius. By default set to \texttt{false}.
\item \texttt{lbfgs\_trust\_radius}: The maximum size of a taken step. By default set to \texttt{0.1}.
\end{itemize}

If you specified \texttt{optimizer: 'steepestdescent'} or \texttt{optimizer: 'sd'}, you can also set the following options:
\begin{itemize}
\item \texttt{sd\_factor}: The scaling factor to be used in the steepest descent algorithm. By default set to \texttt{0.1}.
\end{itemize}

If you specified \texttt{optimizer: 'newtonraphson'} or \texttt{optimizer: 'nr'}, you can also set the following options:
\begin{itemize}
\item \texttt{nr\_trust\_radius}: The trust radius (maximum root mean square) of a taken step. By default set to \texttt{0.5}.
\item \texttt{nr\_svd\_threshold}: The threshold for the singular value decomposition of the Hessian. By default set to
\texttt{1.0e-12}.
\end{itemize}

\subsection{Transition State Optimization}

This task is used to optimize the structure of a given system to a transition state on the potential energy surface. In 
order to carry out this task, specify any of the following in the respective task block: \texttt{type: 'transition\_state\_optimization'}, 
\texttt{type: 'transitionstate\_optimization'}, \texttt{type: 'tsopt'}, or \texttt{type: 'ts'}.

The task works without the specification of any additional settings; the default settings work usually fine. However,
if desired, the following settings can always be set:
\begin{itemize}
\item \texttt{optimizer}: This sets the desired optimization algorithm. You can set \texttt{'bofill'} for Bofill's algorithm\cite{bofill1, bofill2},
or any of \texttt{'eigenvector\_following'}, \texttt{'eigenvectorfollowing'}, \texttt{evf}, or \texttt{ev} for a 
eigenvector following algorithm. By default, it is set to \texttt{'bofill'}.
\item \texttt{convergence\_step\_max\_coefficient}: The convergence threshold for the maximum absolute element of the last step taken.
By default set to \texttt{1.0e-4}.
\item \texttt{convergence\_step\_rms}: The convergence threshold for the root mean square of the last step taken. By default set to 
\texttt{5.0e-4}.
\item \texttt{convergence\_gradient\_max\_coefficient}: The convergence threshold for the maximum absolute element of the gradient. 
By default set to \texttt{5.0e-5}.
\item \texttt{convergence\_gradient\_rms}: The convergence threshold for the root mean square of the gradient. By default set to 
\texttt{1.0e-5}.
\item \texttt{convergence\_delta\_value}: The convergence threshold for the change in the functional value. By default set to
\texttt{1.0e-7}.
\item \texttt{convergence\_max\_iterations}: The maximum number of iterations. By default set to \texttt{100}.
\item \texttt{convergence\_requirement}: The number of criteria that have to converge besides the value criterion 
(\texttt{convergence\_delta\_value}). This has to be between \texttt{0} and \texttt{4}; by default it is set to \texttt{3}.
\end{itemize}

If you specified \texttt{optimizer: 'bofill'}, you can also set the following option:
\begin{itemize}
\item \texttt{bofill\_trust\_radius}: The maximum root mean square of a taken step. By default set to \texttt{0.1}.
\end{itemize}

If you specified \texttt{optimizer: 'eigenvector\_following'}, \texttt{'eigenvectorfollowing'}, \texttt{evf}, or \texttt{ev}, 
you can also set the following option:
\begin{itemize}
\item \texttt{ev\_trust\_radius}: The maximum root mean square of a taken step. By default set to \texttt{0.5}.
\end{itemize}

\subsection{Intrinsic Reaction Coordinate Calculation}

This task is used to an intrinsic reaction coordinate (IRC) calculation. In 
order to carry out this task, specify any of the following in the respective task block: \texttt{type: 'ircopt'}, 
or \texttt{type: 'irc'}. Note that for this task you have to specify two output systems. The first one will contain
the results of the forward IRC calculation while the second on will contain the result of the backward IRC calculation.

You usually want to set the following settings:
\begin{itemize}
\item \texttt{irc\_mode}: This sets the normal mode which should be used for the IRC calculation. By default set to zero
(designates the first normal mode).
\end{itemize}

The task works without the specification of any additional settings; the default settings work usually fine. However,
if desired, the following settings can always be set:
\begin{itemize}
\item \texttt{optimizer}: This sets the desired optimization algorithm. You can set \texttt{'lbfgs'} for the L-BFGS algorithm, and
\texttt{'steepestdescent'} or \texttt{'sd'} for a steepest descent algorithm. By default, it is set to \texttt{'lbfgs'}.
\item \texttt{convergence\_step\_max\_coefficient}: The convergence threshold for the maximum absolute element of the last step taken.
By default set to \texttt{1.0e-4}.
\item \texttt{convergence\_step\_rms}: The convergence threshold for the root mean square of the last step taken. By default set to 
\texttt{5.0e-4}.
\item \texttt{convergence\_gradient\_max\_coefficient}: The convergence threshold for the maximum absolute element of the gradient. 
By default set to \texttt{5.0e-5}.
\item \texttt{convergence\_gradient\_rms}: The convergence threshold for the root mean square of the gradient. By default set to 
\texttt{1.0e-5}.
\item \texttt{convergence\_delta\_value}: The convergence threshold for the change in the functional value. By default set to
\texttt{1.0e-7}.
\item \texttt{convergence\_max\_iterations}: The maximum number of iterations. By default set to \texttt{100}.
\item \texttt{convergence\_requirement}: The number of criteria that have to converge besides the value criterion. This 
must be between \texttt{0} and \texttt{4}; by default it is set to \texttt{3}.
\end{itemize}

If you specified \texttt{optimizer: 'lbfgs'}, you can also set the following options:
\begin{itemize}
\item \texttt{lbfgs\_maxm}: The number of parameters and gradients from previous iterations to keep. By default set to 
\texttt{50}.
\item \texttt{lbfgs\_linesearch}: Whether to use a line search or not. By default set to \texttt{true}.
\item \texttt{lbfgs\_c1}: The first parameter of the Wolfe conditions. This option is only relevant if line search is
used (see above). By default set to \texttt{0.0001}.
\item \texttt{lbfgs\_c2}:  The second parameter of the Wolfe conditions. This option is only relevant if line search is
used (see above). By default set to \texttt{0.9}.
\item \texttt{lbfgs\_step\_length}: The initial step length. By default set to \texttt{1.0}.
\item \texttt{lbfgs\_use\_trust\_radius}: Whether to use the trust radius. By default set to \texttt{false}.
\item \texttt{lbfgs\_trust\_radius}: The maximum size of a taken step. By default set to \texttt{0.1}.
\end{itemize}

If you specified \texttt{optimizer: 'steepestdescent'} or \texttt{optimizer: 'sd'}, you can also set the following options:
\begin{itemize}
\item \texttt{sd\_factor}: The scaling factor to be used in the steepest descent algorithm. By default set to \texttt{0.1}.
\end{itemize}

\subsection{Artificial Force Induced Reaction Calculation}

This task is used in order to do an artificial force induced reaction (AFIR\cite{afir1, afir2}) calculation. In 
order to carry out this task, specify any of the following in the respective task block: \texttt{type: 'afir\_optimization'}, 
\texttt{type: 'afiroptimization'}, \texttt{type: 'afiropt'}, or \texttt{type: 'afir'}.

You usually want to set the following settings:
\begin{itemize}
\item \texttt{afir\_rhs\_list}: This specifies list of indices of atoms to be artificially forced onto or away from those 
in the LHS list (see below). By default, this list is empty. Note that the first atom has the index zero.
\item \texttt{afir\_lhs\_list}: This specifies list of indices of atoms to be artificially forced onto or away from those 
in the RHS list (see above). By default, this list is empty. Note that the first atom has the index zero.
\end{itemize}

The task works without the specification of any additional settings; the default settings work usually fine. However,
if desired, the following settings can always be set:
\begin{itemize}
\item \texttt{afir\_weak\_forces}: This activates an additional, weakly attractive force applied to all atom pairs. By 
default set to \texttt{false}.
\item \texttt{afir\_attractive}: Specifies whether the artificial force is attractive or repulsive. By default set to
\texttt{true}, which means that the force is attractive.
\item \texttt{afir\_energy\_allowance}: The maximum amount of energy to be added by the artifical force, in kJ/mol.
By default set to \texttt{1000}.
\item \texttt{afir\_phase\_in}: The number of steps over which the full attractive force is gradually applied. By default
set to \texttt{100}.
\item \texttt{afir\_transform\_coordinates}: Whether to transform the coordinates from a Cartesian basis into an internal 
space. By default set to \texttt{true}.
\item \texttt{optimizer}: This sets the desired optimization algorithm. You can set \texttt{'lbfgs'} for the L-BFGS algorithm, and
\texttt{'steepestdescent'} or \texttt{'sd'} for a steepest descent algorithm. By default, it is set to \texttt{'lbfgs'}.
\item \texttt{convergence\_step\_max\_coefficient}: The convergence threshold for the maximum absolute element of the last step taken.
By default set to \texttt{1.0e-4}.
\item \texttt{convergence\_step\_rms}: The convergence threshold for the root mean square of the last step taken. By default set to 
\texttt{5.0e-4}.
\item \texttt{convergence\_gradient\_max\_coefficient}: The convergence threshold for the maximum absolute element of the gradient. 
By default set to \texttt{5.0e-5}.
\item \texttt{convergence\_gradient\_rms}: The convergence threshold for the root mean square of the gradient. By default set to 
\texttt{1.0e-5}.
\item \texttt{convergence\_delta\_value}: The convergence threshold for the change in the functional value. By default set to
\texttt{1.0e-7}.
\item \texttt{convergence\_max\_iterations}: The maximum number of iterations. By default set to \texttt{100}.
\item \texttt{convergence\_requirement}: The number of criteria that have to converge besides the value criterion. This 
has to be between \texttt{0} and \texttt{4}; by default it is set to \texttt{3}.
\end{itemize}

If you specified \texttt{optimizer: 'lbfgs'}, you can also set the following options:
\begin{itemize}
\item \texttt{lbfgs\_maxm}: The number of parameters and gradients from previous iterations to keep. By default set to 
\texttt{50}.
\item \texttt{lbfgs\_linesearch}: Whether to use a line search or not. By default set to \texttt{true}.
\item \texttt{lbfgs\_c1}: The first parameter of the Wolfe conditions. This option is only relevant if line search is
used (see above). By default set to \texttt{0.0001}.
\item \texttt{lbfgs\_c2}:  The second parameter of the Wolfe conditions. This option is only relevant if line search is
used (see above). By default set to \texttt{0.9}.
\item \texttt{lbfgs\_step\_length}: The initial step length. By default set to \texttt{1.0}.
\item \texttt{lbfgs\_use\_trust\_radius}: Whether to use the trust radius. By default set to \texttt{false}.
\item \texttt{lbfgs\_trust\_radius}: The maximum size of a taken step. By default set to \texttt{0.1}.
\end{itemize}

If you specified \texttt{optimizer: 'steepestdescent'} or \texttt{optimizer: 'sd'}, you can also set the following options:
\begin{itemize}
\item \texttt{sd\_factor}: The scaling factor to be used in the steepest descent algorithm. By default set to \texttt{0.1}.
\end{itemize}

\section{Task Chaining}
\label{sec:task_chaining}

You can specify multiple tasks to be executed after each other. Tasks are processed in the order in which they are given in 
the input file. For example, the following input file would first carry out a structure optimization, and then calculate
the vibrational frequencies of the optimized structure:

\begin{verbatim}
systems:
  - name: 'water'
    path: 'h2o.xyz'
    program: 'Sparrow'
    method: 'PM6'

tasks:
  - type: 'geoopt'
    input: ['water']
    output: ['water_opt']
  - type: 'hessian'
    input: ['water_opt']
\end{verbatim}



\chapter{Using the Python Library}

\textsc{ReaDuct} provides Python bindings such that all functionality of \textsc{ReaDuct} can be accessed also via the
Python programming language. In order to build the Python bindings, you need to specify \texttt{-DSCINE\_BUILD\_PYTHON\_BINDINGS=ON}
when running cmake (see also chapter~\nameref{ch:installation}).

In order to use the Python bindings, you need to specify the path to the Python library in the environment variable
\texttt{PYTHONPATH}, \textit{e.g.}, you have to run the command
\begin{verbatim}
export PYTHONPATH=$PYTHONPATH:<source code directory>/install/lib
\end{verbatim}
Now, you can simply import the library and use it as any other Python library. For example, in order to carry out a 
structure optimization, you could use the following Python script:
\begin{verbatim}
import scine_readuct

system1 = scine_readuct.load_system('h2o.xyz', 'PM6', program='Sparrow', 
                                    molecular_charge=0, spin_multiplicity=1)

systems = {}
systems['water'] = system1

task1 = scine_readuct.run_opt_task(systems, ['water'], output=['water_opt'], 
                                   optimizer='lbfgs')
systems['water_opt'].positions
\end{verbatim}
Note that we use a dictionary called ``systems'' to store all systems we deal with in one central data
structure. As second argument, the structure optimization task accepts a list of the systems which should be optimized,
\textit{i.e.}, the dictionary ``systems'' can contain more systems but these will not be optimized (all other tasks
work with the same concept). The output system(s) will be automatically added to the systems dictionary.

A detailed list of all the functions provided by the \textsc{ReaDuct} Python library can be found by running
\begin{verbatim}
import scine_readuct

help(scine_readuct)
\end{verbatim}



\chapter{Extensions Planned in Future Releases}
\begin{itemize}
\item Interfaces to other quantum chemical packages such as \textsc{Gaussian} and \textsc{Serenity}\cite{serenity}
\item Implementation of B-Spline optimization of transition state structures\cite{bsplines}
\end{itemize}


\chapter{Important References}

Please consult the following references for more details on \textsc{ReaDuct}.
We kindly ask you to cite the following reference in any publication of results obtained with \textsc{ReaDuct}.
\vspace{1.0cm}

A.~C.~Vaucher, M.~Reiher \href{https://pubs.acs.org/doi/10.1021/acs.jctc.8b00169}{"Minimum Energy Paths and Transition States by Curve Optimization"}, \textit{J.~Chem.~Theory Comput.}, \textbf{2018}, \textit{16}, 3091.



%%
% The back matter contains appendices, bibliographies, indices, glossaries, etc.

\backmatter

\bibliography{references}
\bibliographystyle{achemso}

%\printindex

\end{document}
